% EXPERIENCIA PROFESIONAL y DOCENCIA
\section{\textsc{Experience}}
\cventry{Aug 2020  \\ --}{HPC and Cloud Lead Engineer}{Scientific Computing Platform (SCP)}{Astrazeneca PLC}{}{
As an Engineer Lead in the SCP team, I work with a highly skilled team of people distributed worldwide to deliver, evolve, and support all critical services and capabilities offered by the Scientific Computing Platform. 
%Responsibilities include, but are not limited to:
% \begin{itemize}
% \item Coordinated projects by assembling teams and assigning duties.
% \item Identified the training needs of other team members.
% \item Established effective communications with all engineering teams.
% \item Provided technical oversight on various projects and offered input.
% \item Reviewed quality and testing procedures and suggested updates.
% \item Interfaced with other department leads.
% \end{itemize}
}
\cventry{Jan 2020  \\ Aug 2020}{High Performance Computing Engineer}{Scientific Computing Platform (SCP)}{Astrazeneca PLC}{}
{
The Scientific Computing Platform is AstraZeneca’s state-of-the-art computing environment that provides building capabilities and services. We tackle this challenge by offering three main platforms; a classical InfiniBandSlurm HPC cluster, an Openstack private cloud, as well as Public Cloud for elasticity and scale. To exploit these resource pools most efficiently, our team is taking advantage of robust DevOps tooling, cloud-native technologies and an Agile mindset.
\medskip\\ 
\textbf{Keywords:} Linux, High-Performance Computing, Cloud Computing (GCP, AWS), OpenStack, CEPH Storage, Infiniband, SLURM, GPFS, NiceDCV, JupyterHub, Kubernetes, Hashistack (Consul, Vault, Terraform, Nomad), Ansible, Molecule, TestInfra, Python, Bash\medskip}

\cventry{Jun 2019  \\ Jan 2020}{Cloud Engineer}{Technical Services Cluster (TSC)}{European Bioinformatics Institute (EMBL-EBI)}{}
{
As a Cloud Engineer on the Virtualisation and Cloud team, my main goal was to improve existing processes and platforms and implement new automation techniques. Additionally, I worked architecting and developing the next version of the Embassy Cloud service (IaaS), a Private Cloud platform (OpenStack based). Moreover, during this period, we successfully delivered a new Container as a Service (CaaS) integrated with our internal Service Desk platform to set up a fully automated end-to-end self-service. Lastly, I supported the adoption of the SCRUM framework and Agile mindset within the team, accelerating the delivery and improving the flow of value.
\medskip\\ 
\textbf{Keywords:}  Linux, Cloud Computing, OpenStack, VMWare, CEPH Storage, Packer, Terraform, Vault, Ansible, Kubernetes, Gitlab, Rundeck, Prometheus \medskip}

\cventry{Jun 2018  \\ Jun 2019}{Research Computing Solutions Specialist}{HPC Services (HPCS)}{University of Cambridge}{}
{
I joined the HPC Platforms team with the primary goal of transforming HPC from its traditional static old fashioned way into a more modern and agile by applying a DevOps approach. During this year, as a team, we achieve to deliver a production IaaS service (OpenStack based) and certify this service under the ISO-27001/2 regulation. As a result, we offered a customised and Secure On-demand High-Performace Computing platform service (PaaS) on top of this framework.
\medskip\\ 
\textbf{Keywords:}  Linux, Cloud Computing, OpenStack, CEPH Storage, Packer, Terraform, Ansible, Docker, Python, RestAPI, Kubernetes, Prometheus, ElasticSearch \medskip}

\cventry{Nov 2016  \\ Jun 2018}{Head of Systems Architecture}
{Mind the Byte}{Parc Científic de Barcelona (UB)}{}
{
I joined Mind the Byte as a Head of Systems Architecture to provide new methods to improve Scalability, Reliability, and Security for the next evolution of their pay-per-use SaaS platform. That includes providing a modern DevOps approach to deliver value with maximum speed, functionality and innovation. The main goal was to achieve a real continuous software delivery environment and reduce the complexity to manage a complex production environment while improving resiliency and availability.
\medskip\\ 
\textbf{Keywords:}  Cloud Computing, AWS (EC2, VPC, Lambda, S3, Cloudfront, Cloudformation), Packer, Terraform, Ansible, Vagrant, Docker, SLURM, NFS, Python, Nagios, Prometheus, Grafana, ELK, Alert Manager, MongoDB\medskip}

\cventry{Jun 2013  \\ Nov 2016}{Research Assistant, HPC System Administrator}
{Department of Computer Architecture and Operating Systems (CAOS)}{Universitat Aut\`onoma de Barcelona (UAB) | Centre for Research in Agricultural Genomics (CRAG)}{}
{
As a result of a collaboration agreement between UAB and CRAG, I joined the Bioinformatics Core Unit, providing solutions in many aspects, such as HPC system administration, but also the development of parallel applications (MPI, OpenMP). Everyday tasks include network administration, accounting, monitoring, reporting, system administration of storage services, and procedure automation.
\medskip\\ 
\textbf{Keywords:} Galaxy project, Spacewalk, Modules Environment, CUDA, MPI, OpenMP, SLURM, NFS, Lustre, Red Hat, CentOS, VMware VSphere, Docker, R, C++, Bash, Perl, Nagios, Ganglia, ELK
\medskip}

\cventry{Feb 2012 \\  Feb 2013}{Research support, HPC system administrator}{Bioinformatic Core Unit}{Centre for Research in Agricultural Genomics (CRAG)}{}
{
I joined CRAG as a technical expert to help design and install the first High-performance Computing platform. Along this year, we successfully installed and configured a completed HPC cluster and related services like helpdesk, provisioning, and monitoring.
\medskip\\ 
\textbf{Keywords:} Spacewalk, Cobbler, Puppet, Modules Environment, SLURM, NFS, Lustre, Red Hat, CentOS, Cluster VMware VSphere, Nagios, Ganglia
\medskip}

\cventry{Jul 2007 \\ Jun 2013}{Technical Research Assistant}
{Department of Computer Architecture and Operating Systems (CAOS)}{Universitat Aut\`onoma de Barcelona (UAB)}{}
{
I worked as a System Administrator with different High-performance Computing environments, providing services at all levels: installation, troubleshooting, performance analysis, tuning, and daily-based tasks like administering, monitoring, and support to experienced high-level users (PhD students in HPC).
\medskip\\ 
\textbf{Keywords:} XCAT, Cobbler/Puppet, Modules Environment, Checkpoint libraries, CUDA, MPI, OpenMP, SGE, PBS, NFS, Debian, Suse 10, Red Hat, Cytrix, VMware ESXi, Nagios, Ganglia
\medskip}

\cventry{Sept 2006 \\ Jul 2007}{Technical Research Assistant}{Escola d'Enginyeria}{Universitat Aut\`onoma de Barcelona (UAB)}{Technical Support and Systems Administration, \emph{Grid Computing}}
{
I helped in the development of the initial version of the CrossBroker Grid Scheduler, a job management system for interactive and parallel jobs included in the Interactive European Grid (Int.eu.grid) project. I gained experience in different Grid-based systems ( EGEE2, Globus, and Condor), and local resource management systems like Sun Grid Engine (SGE), Torque-Maui, and Portable Batch System (PBS). 
\medskip\\ 
\textbf{Keywords:} EGEE2, Globus, Condor, SGE, NFS, Debian, Xen, Bash, Perl
\medskip}

\cventry{Jun 2005 \\ Sept 2006}{Analyst programmer}{Agresso Spain, S.L.}{Department of Computer Information Systems}{Web applications development and Systems Administration}{}{}


\subsection{\textsl{Teaching}}
\cventry{Sept 2014 Aug 2015}{Assistant Lecturer}{Autonomous University of Barcelona (UAB)}{
Teacher assistant of undergraduate Computer Engineering courses: Distributed Systems (Hadoop framework), and  High-Performance Computing (MPI, GPU, and OpenMP programming)
}{}{}

\cventry{2011}{Managing, and configuring a High-Performance Computing Cluster (HPC)}{Universidad Nacional de San Luís (UNSL)}{Argentina}{
Invited as Technical Expert in High-Performance Computing to teach the practical part of the course: How to setup and configure an HPC cluster.}{}
\closesection